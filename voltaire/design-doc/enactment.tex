\newpage
\section{Proposal Enactment}
\label{sect:enactment}

When the stated deadline for a proposal is reached, then it is considered for enactment.  Proposals are never considered for enactment
before the deadline.  A proposal that has been accepted by passing a voting threshold (and also, for a protocol version upgrade, the
endorsement threshold) is automatically enacted.  Enactment involves each protocol parameter being updated to the new value that is specified
in the proposal.  Changes take effect at the start of the following epoch.  A protocol version upgrade can also update other parameters as part of
the proposal, and may even change the set of parameters that can be updated.

\subsection{Protocol version changes (``Hard Forks'')}

Protocol version upgrades never decrement the protocol version.  They must increment either the major or minor version.  If the major version is
incremented, then the minor version may be set to any required value (usually this will be zero).  Software upgrades must accept all legitimate prior protocol versions.
When a protocol version proposal is enacted, then all nodes will synchronise to the new protocol version using the ``hard fork combinator''~\cite{hardfork-combinator}.
Block production and verification will continue at the start of the next epoch using the rules that are in force for the new protocol version.
The only possible conflict is where two or more properly approved and endorsed proposals increment both the major and the minor version number in the same epoch.  In this case, the major version change will always succeed, regardless of the order in which they are enacted.

\subsection{Prioritisation of Enactment}

If multiple update proposals are to enacted within a single epoch, then they are enacted strictly in the order that they were submitted.
This creates a deterministic order that does not need to consider when proposals achieve a majority vote, for example.
It allows proposals to be completely or partially overridden without needing an explicit cancellation option.
This approach assumes, of course, that the submitter group acts honestly and in the best interest of the protocol.
The approach does not allow contingent/conditionals proposals.  If needed, this can be handled by, for example, submitting two separate proposals that are enacted
in different epochs, where the second proposal is submitted only if the first proposal succeeds.

\subsection{Central Funds Transfers (``MIRs'')}

Several proposals for funds transfers may be enacted in a single epoch.  These proposals are enacted the in the order that they were submitted, with all ledger and
accounting rules in force.  In particular, it is not possible for the treasury, reserves or any private address to even temporarily have a negative balance.  This may affect
the ordering of transfers in some edge cases.
