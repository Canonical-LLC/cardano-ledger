\newpage
\section{Vote Delegation (On-Chain)}
\label{sect:delegation}

All ada holders will have on-chain voting rights.  These rights are in direct proportion to the exact number of lovelace that they hold.
They may delegate their voting rights to any registered delegate, who is expected to act faithfully on their behalf.

\subsection{Delegate Address Registration/De-Registration}

Delegates must register on-chain.\khcomment{Do we need a certificate?  I think they only need an address, right?}  Once registered, ada holders may delegate their
vote to any registered delegate by submitting a vote delegation transaction that includes the registered delegate address.\khcomment{How do voters know about the available delegates?}
Ada holders may change this delegation at any point.  When votes are tallied, the most recent delegation is considered (in effect, a snapshot is taken for the tally).
A deposit is taken when a delegate registers.  Delegates may choose to de-register at any time by submitting a de-registration transaction.  The de-registration takes effect
at the specified point.  A protocol paramter governs the least and greatest notice that must be given for de-registration.  When a delegate de-registers, their deposit is returned,
the address becomes void, and any vote delegation to that address no longer has any effect.

\subsection{Voting Indirection}

The current ledger design separates spending and block production delegation, associating two keys with a single address.

%\begin{figure}[h!]
\begin{center}
  \begin{tabular}{||l|l||}
\hline\hline
  payment key & spending and withdrawal rights \\\hline
  stake key & block production delegation rights \\\hline
  \hline\hline
  \end{tabular}
\end{center}
%  \caption{Address and key rights in Cardano: Status Quo}
%\end{figure}

The new design adds an additional key for vote delegatation.

% \begin{figure}[h!]
\begin{center}
  \begin{tabular}{||l|l||}
\hline\hline
  payment key & spending and withdrawal rights \\\hline
  stake key & block production delegation rights \\\hline
  vote key & vote delegation rights (new) \\
\hline\hline
\end{tabular}
\end{center}
%  \caption{Address and key rights in Cardano: New Design}
%\end{figure}

It is necessary to register both block and vote delegation intentions on-chain.  This may be done in a single transaction or independently.
The design allows independent vote and block production delegation, and also allows those rights to be passed to a third party (by providing them
with the corresponding private keys). \khcomment{Make sure this is clear and consistent with the current stake key approach.} \khcomment{Confirm that a single transaction can be used.}
The private key can be passed to another party to allow for transfers of delegation rights, but there is no way to independently revoke any key, or to remove
delegation capability without revoking the corresponding key.  \khcomment{Confirm this.}

\begin{figure*}[h]
  \caption{Examples of Indirection}
\end{figure*}
