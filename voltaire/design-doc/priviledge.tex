\section{Comparison with the approach taken by the Priviledge Project}

The EU Priviledge project aims to produce a blockchain-agnostic fully decentralised voting system.
Its design is described in~\ref{Priviledge}, and has been used to inform the design that is described here
(PUP).  The key differences between the PUP mechanism and the one that has been produced for the EU Priviledge project are:

\begin{tabular}{||p{3in}|p{3in}||}
  \hline\hline
  \textbf{Priviledge} & \textbf{PUP}
  \\\hline
  Completely on-chain & Uses off-chain as well as on-chain \\\hline
  Does not restrict proposal submission & Restricts on-chain submission to a specific group \\\hline
  Proposals are tracked throughout & Proposals are tracked once they are on-chain \\\hline
  Does not identify actor groups & Clearly identifies actor groups \\\hline
  Supports vote abstention & Treats abstention as rejection \\\hline
  Allows multiple voting rounds & Allows only one on-chain voting round \\\hline
  Allows  vote stake snapshots to be taken at arbitrary points & Takes vote snapshots at fixed points \\\hline
  Explicit prioritisation & Submission time-based prioritisation \\\hline
  Includes software upgrades & Does not consider software upgrades, except where these are protocol version changes \\\hline
  Does not handle central funds transfers & Handles central funds transfers \\\hline
  Does not support vote delegation & Supports vote delegation \\\hline
  Requires version numbering & No version numbering \\\hline
  Single proposal enactment at an epoch boundary\khcomment{From memory.  Check this.}  & Multiple proposal enactment at each epoch boundary \\\hline
  %   Does not include specific protocol parameters & Includes protocol parameter updates \\\hline
  %% Not sure what I meant by that?  KH
  \hline
\end{tabular}

Most of these changes reflect differences in requirements, or a desire to make
the new mechanism generally consistent with the existing update mechanism.
Others are simplifications that are aimed at improving peformance (e.g. vote
stake snapshots) or reducing development and testing time (e.g. version numbering, time-based prioritisation).

\khcomment{First draft.  Each of these points could be expanded on, if required.}
