\pagebreak
\section{Main Groups Involved in Enacting new Proposals}
\label{sect:groups}

We can identify a number of distinct groups that may be involved in proposal submission, discussion and enactment.

\vspace{1em}
\begin{tabular}{||l|l|l||}
  \hline \hline
  \textbf{Group} & \textbf{How Involved} & \textbf{On-/Off-Chain?}\\
  \hline
  General Ada Holders  & Make new informal proposals. & Both \\
                       & Discuss and vote on proposals off-chain. & \\
  			               & Delegate on-chain voting rights. & \\
  \hline
  Executive Team  & Prioritise proposals. & Off-chain \\
                       & Manage off-chain votes. & \\
                       & Manage proposal discussion. & \\
  			               & Ensure that proposals are implemented. & \\
  			               & Ensure that proposals are submitted on-chain. & \\
  \hline
  Security Experts			       & Advise on Security Aspects of proposals. & Off-chain \\
                               & Track security issues and suggest fixes. & \\
  \hline
  Other Expert Groups			       & Provide feedback on proposals. & Off-chain \\
  \hline
  Proposal Editors			   & Ensure that proposals are documented & Off-chain \\
                     & in the required way. &  \\
  \hline
  Proposal Implementors			   & Produce formal proposals/software. & Off-chain\\
  \hline
  Submitters				   & Submit formal proposals on-chain. & On-chain \\
  \hline
  Delegates				   & Vote on acceptance of on-chain proposals. & On-chain \\
  \hline
  Endorsers			   & Endorse readiness for protocol version upgrade. & On-chain\\
%   \hline
%   Electors			       & Nominate submitters. & Either \\
  \hline \hline
\end{tabular}

\subsection{General Ada Holders (Off-Chain and On-Chain)}

General Ada holders may make informal proposals and discuss them using an off-chain mechanism.  They vote on the acceptance of these proposals
both on-chain and off-chain, in proportion to the ada that they hold.  They may delegate their stake for voting purposes to on-chain representatives,
who will vote on their behalf for or against submitted proposals.  The vote delegation may be changed at any time prior to the voting deadline.
The final delegation is the one that is considered for tallying purposes.
% \pkcomment{We should clarify whether this includes changing the delegation \emph{after} delegates have voted. I think it would be good to allow this}\khcomment{Yes, the design allows that up to the vote deadline}.

\subsection{Executive Team (Off-Chain)}

The Executive Team is responsible for managing proposals, prioritising them for discussion and voting, monitoring the voting process and the
voting outcome, ensuring that the proposal is properly implemented and submitted on-chain, and reporting back on proposal enactment.
It is also responsible for ensuring that the correct process is followed for the type of proposal.  It may, eventually, be possible to automate part or all of this process.

\subsection{Security Experts (Off-Chain)}

Security Experts evaluate proposals, provide advice on possible attack vectors, and suggest improvements to mitigate security risks.
Their input may not be necessary on some simple parameter changes, but is always necessary for protocol version changes.
They may also evaluate security threats and may, in extremis, produce proposals that bypass some of the usual governance processes
(eg where open discussion could lead to a major risk of loss to the blockchain).  This creates a risk that needs to be
considered as part of the design of an effective governance system.

\subsection{Other Expert Groups (Off-Chain)}

Other Expert Groups may be called on to provide advice or reports to inform general ada holders or other groups that are involved in the
governance process.  This could include performance issues, technical blockchain expertise, economic advice, advice on implementation etc.
This advice is critical if rational decisions are to be taken on technical issues.

\subsection{Proposal Editors (Off-Chain)}

Proposal Editors\footnote{Currently CIP Editors, where CIP stands for Cardano Improvement Proposal.} ensure that proposals are documented correctly, and that they are recorded in a
permanent repository.
This provides a persistent record of the rationale for each proposal, and allows traceability between informal off-chain
proposals and the formal on-chain proposal.  This process may include providing hashes or other information to support traceability,
ensuring that proposals are valid, that conflicts between proposals are identified, and that appropriate technical input and
guidance is obtained where necessary.

\subsection{Proposal Implementors (Off-Chain)}

Proposal Implementors transform accepted proposals that have been voted on by ada holders into formal proposals that can be submitted on chain.
They are responsible for providing necessary documentation, and for compliance with the required processes.  This may include producing an acceptable formal on-chain proposal,
updating the off-chain parameter documentation,
and meeting any necessary verification/traceability requirements.  For simple parameter updates or funds transfers, the implementation may be in the form of a single
source file\footnote{Currently in JSON format.} that fully describes the proposal. Protocol version upgrades will also require the corresponding software implementation.

\subsection{Submitters (On-Chain)}

Submitters are permitted to submit proposals on-chain.  Each proposal must be signed by a quorum of submitters\footnote{It would be possible to split the roles into distinct submitter and counter-signatory groups, if preferred.},
who verify the submitted proposal, checking that it is in the correct format, conforms to the original proposal, and meets other process requirements that are laid down.
% Submitters are appointed by electors.
Depending on the governance design, the submitter group might % be restricted to a few members,
be an elected or appointed group, be limited by ada holding, or be limited to endorsers, delegates or another group.
It is important for dececentralisation that the group is not restricted to too few members.
However, to avoid overwhelming voters by ``flooding'' the system with voting choices, it is also important to limit the number of proposals that can
be submitted, which is most easily done by limiting the number of submitters.  This implies that the required quorum should be a significant percentage
of the submitter group.
\emph{Since this group controls what proposals are ultimately voted on, care needs to be taken over its size, composition and accountability.  It is important that there is an ``honest'' majority of submitters,
so that proposals are submitted in accordance with prior approvals, and that proper care is taken over proposal verification.  It is also important that the group is ``legitimate''
and can be changed through some proper process.  This avoids a small group gaining control.  Collectively, the group has the power of veto -- either by refusing to submit a proposal, or by
refusing to counter-sign a proposal.  This helps ensure the quality of proposals and limits the volume that need to be considered by the voter group.  Where necessary, the group is also responsible for ordering
proposals so that a sensible enactment order is applied, or grouping proposals so that a single vote can cover a set of related proposals (e.g. funds transfers from some Catalyst round).}

\subsection{Delegates (On-Chain)}

Delegates vote on whether an on-chain proposal should proceed.  The weight of
their vote is in proportion to the stake that has been delegated to them by
normal ada holders, including themselves, as determined at the point when a vote is tallied.
They are expected to be responsible
experts, to represent those who have chosen to delegate vote to them, and will
be respected community members.  Although in many cases delegates may also be
stake pool operators, the separation of vote delegation from delegation for
block production separates politics from economics.  Ada holders may choose to
delegate ada to good block producers in an apolitical way, and to support
delegates that represent their political opinions without considering their abilities as
block producers.  Delegates must track any discussions and ensure that the proposals meet the required
security and other concerns.

\subsection{Endorsers (On-Chain)}

The role of an endorser is to confirm that they are ready to upgrade to a new
version of the Cardano protocol.  All stake pools act as endorsers.  In line with
the principle of proof-of-stake, the weight of their endorsement is proportional
to the stake that they control.  In order to ensure that the blockchain will
make progress, and that the main chain will not split, a sufficient stake
threshold of endorsement is required for a proposal to be enacted.
The design assumes that endorsement will be automated by recording the stake that
is associated with upgraded pools as they produce blocks (Section~\ref{sect:endorsement}).


%% \subsection{Electors (Off-chain or On-Chain)}

%% Electors determine who belongs to the submitter group.  The composition of this group is to be determined.  It might be formed by direct election off-chain\pkcomment{If it is an off-chain election, how do we ensure the correct result is recorded on the chain?}, by
%% nomination from an agreed group, by submission of an on-chain proposal.  In some systems, it may even be unnecessary.
%% One possibility is to have a self-perpetuating and renewing submitter group, so that submitters nominate and sign for additional members, or remove existing members.
%% \khcomment{This needs to be agreed.}
