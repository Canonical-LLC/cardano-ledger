\section{Comparison with the approach taken by the Priviledge Project}
\label{sect:priviledge}

The EU PriviLedge project aims to produce a blockchain-agnostic fully decentralised voting system.
Its design is described in~\ref{Priviledge}, and has been used to inform the design that is described here.
The PUP design limits itself to on-chain submission, voting and enactment of proposals as they need to be
adapted for use in the Cardano blockchain, leaving the ideation, discussion, and general voting phases open to off-chain solutions.
In contrast, the Priviledge design also covers on-chain ideation and voting.
As already described, it would be possible to provide on-chain ideation and voting phases in future, adapted from the PriviLedge
mechanisms to take into account the specific Cardano requirements in terms of functionality, efficiency and security.

\subsection{Key Differences between the PUP and PriviLedge Mechanisms}

The key differences between the PUP mechanism and the one that has been produced for the EU PriviLedge project are shown in the table below
(this table has been revised to reflect the current PriviLedge implementation for Cardano -- additional changes could be made to the PriviLedge design
that would create further alignment).

\begin{tabular}{||p{3in}|p{3in}||}
  \hline\hline
  \textbf{Priviledge} & \textbf{PUP}
  \\\hline
  Completely on-chain & Uses off-chain processes as well as on-chain ones \\\hline
%  Proposal implementations must be completed before any voting can commence\khcomment{Confirm this}. & Proposal implementations must be completed before being submitted on-chain \\\hline
%  New end-to-end mechanism & Conservative automation of existing update mechanism \\\hline
  Does not restrict proposal submission tof specific individuals\footnote{It would be possible to do this, but it could change some of the research results (e.g. for chain liveness)} & Restricts on-chain submission to a specific group\footnote{This aims to avoid ``flooding''/denial-of-service, but assumes an honest submitter group.}\\\hline
  Proposals are tracked throughout & Proposals are tracked once they are on-chain, mechanisms are provided to tie enactment to off-chain/on-chain proposals \\\hline
  Each proposal is independent & A single off-chain proposal may initiate multiple on-chain proposals for enactment \\\hline
  Security-related proposals follow the standard process  & Security-related proposals may bypass some off-chain processes \\\hline
  A single vote may only be used for one active proposal\footnote{This is intended to ensure that votes are consistent (so that a voter does not vote for one proposal, and also for a contradictory proposal, for example).  However, this could create problems with
    securing sufficient vote to pass a threshold.}
    & A single vote may be used for any active proposals \\\hline
  Does not identify different actor groups\footnote{Submitters are currently drawn from the full voting group.} & Clearly identifies different actor groups and privileges: submitters, delegates etc. \\\hline
  Supports explicit vote abstention & Treats abstention as rejection\footnote{This is a minor simplification, but it reduces the complexity of vote tallying and what needs to be recorded on chain.} \\\hline
  Allows votes to be removed prior to tally & Allows votes and delegations to be \emph{changed} prior to tally\footnote{The effect is equivalent, except that in the Priviledge design, the vote could be reassigned to another proposal.  This is not a concern with the PUP design, since it is anyway possible to vote on multiple proposals.} \\\hline
  Allows multiple voting rounds for each proposal & Only allows a single on-chain voting round for a proposal\footnote{This has the advantage of removing proposals that are not approved, so reducing the number of choices that are presented to the voter group, and avoiding chain ``clogging'' over time.} \\\hline
  %  Allows  vote stake snapshots to be taken at arbitrary points & Takes vote snapshots at fixed points \\\hline
  % The implementation has changed this design
  \hline
% \end{tabular}
%
%\begin{tabular}{||p{3in}|p{3in}||}
%  \hline\hline
%  \textbf{Priviledge} & \textbf{PUP}
%  \\\hline
  Explicit proposal prioritisation -- order of enactment can be specified  & Submission-based prioritisation -- proposals are enacted in order of submission \\\hline
  Proposals can be cancelled & Proposals cannot be cancelled once they are approved and endorsed \\\hline
  Includes software upgrades (not just protocol upgrades) & Does not consider software upgrades, except indirectly via protocol version changes \\\hline
  Does not handle central funds transfers & Handles central funds transfers \\\hline
%  Does not support vote delegation\footnote{New work is in progress.} & Supports vote delegation \\\hline
%  Does not consider Cardano blockchain stability windows & Explicitly considers Cardano blockchain stability windows \\\hline
  Requires version numbering in each proposal & No version numbering required  \\\hline
  Single proposal enactment at an epoch boundary  & Multiple proposal enactment at each epoch boundary \\\hline
  \hline
\end{tabular}

Most of these differences reflect either differences in stated user requirements, or a desire to make
the PUP mechanism generally consistent with the existing update mechanism.
Other differences are simplifications that are aimed at improving peformance or security (e.g. vote
stake snapshots, restricting submission) or reducing development and testing time (e.g. version numbering, time-based prioritisation).
Most of the Priviledge mechanisms could be added at a later date, to create a full on-chain ideation, approval and enactment process.
The proposal submission process could then be adapted so that proposals were automatically carried forwards from previous Priviledge stages,
using the proposal identifier to link approved proposals with those that are to be endorsed for enactment.
