\pagebreak
\section{Efficiency Concerns}
\label{sect:efficiency}

This appendix considers possible efficiency concerns.  Overall, the design will lead to some (limited) increase in memory
usage (proportional to the number of registered vote addresses), plus some additional transaction costs.
Proposal submission and enactment should carry minimal additional costs compared with the current manual process.
Additional costs are associated with vote snapshots, registration, vote delegation, and vote tallying.
These costs are bounded by the product of the number of vote delegations and the number of active proposals.
Overall, the costs are manageable and proportionate, provided that there are not a large number of active proposals.
It may be advisable, however, to group a set of proposals under a single vote.

\subsection{Proposals and Proposal Submission}

A single proposal needs to be submitted with appropriate signatures.  The cost of this submission should be no greater than
a normal transaction submission.  Proposals need to be stored for voting and enactment.  An in-memory hash is sufficient
for each proposal up to the point of enactment.  This hash could be recorded on-chain.

Assuming that the majority of the submitter group is honest, then the number of proposals that are under consideration
at any point in time should be small and finite.  By setting time limits on enactment, this number can be controlled.

Since there is likely to be at most only one or two active protocol upgrade proposals, and a small number of active parameter
changes, the most significant pressure is likely to come from funds transfer proposals (``MIRs'').  Because only a relatively
small number of transfers can be included in a single proposal\footnote{Around 100-200}\todo{check this number},
a single high-level funding decision could, in principle, result in several on-chain proposals.
Submitters should take care over the number of such proposals, and consider submitting them over multiple epochs, if possible.
If this is expected to be common, then it may be sensible for a single vote to activate a group of proposals.
This will reduce the number of vote transactions, as well as reducing the risk of voter fatigue (or only part of a group
being authorised).


% It is necessary to avoid ``flooding'', where many junk proposals are subm.

\subsection{Vote Snapshots}

It is necessary to take vote snapshots before tallying votes.  Since different
credentials are used for vote and stake delegation, this is in addition to the
usual stake snapshot.  Unlike stake delegation, only a single active snapshot is required.
It is not necessary to calculate this during the epoch transition, provided that it is available
at the point at which any votes are tallied.  It follows that the snapshot could be calculated
incrementally, if desired.  This would reduce calculation pressure.  There are approximately 420,000
registered stake addresses\footnote{2021-04-09}.  Storing a similar number of vote addresses would require approx. 15MB of
memory\khcomment{assuming 28 bytes for an address hash, 4 bytes for a value, and 4 bytes for a header word.}.

\subsection{Vote Delegation}

A one-off cost (time/ada/chain growth) is paid for registering each vote delegation
key.  This will be similar to the cost that is paid for e.g. registering a stake key.
At most one key must be registered for each active payment credential.
Certificates are registered on-chain.  There is no direct impact on the size of
the UTxO.  A similar cost is paid each time a vote delegation is made or changed.

\subsection{Delegates and Voting}

A one-off cost (time/ada/chain growth) is paid for registering each delegate.
This will be similar to the cost that is paid for e.g. registering a stake key.
Delegates must also submit votes.  There will be one vote transaction per delegate per proposal.
Assuming that there is a small number of active proposals and the number of delegates is
significantly less than the number of registered voters, we can estimate the typical number of
votes per epoch to be less than 10,000. This would be covered by around 500s of total transaction time.

\subsection{Vote Tallying}

Tallying the vote involves summing the delegations that have been made in favour of each active proposal.  Since delegation can be
changed up to the point where a tally is taken, this must be done after the tally point and prior to enactment.
In the worst case, each tally involves adding the stake that is associated each delegated vote (potentially \emph{420,000}
additions).  This can, in principle, be parallelised and provided that sufficient time is available between the tally
and the proposal enactment, can be carried out incrementally.  Limiting the number of active proposals will also
limit the total computational time.
The total possible delegation can be pre-calculated from the snapshot, so should not add significantly to performance costs.

\subsection{Endorsement}

Automatic endorsement carries minimal time overhead.  As each block is produced, the corresponding  pool identifier needs to be
recorded.  The total stake that is prepared to upgrade needs to be calculated prior to the endorsement.  Assuming that there are
approximately \emph{2k} block producing pools and \emph{k} is 2,500 or less, this will involve a maximum of 2,500 additions,
which should have negligible performance impact.

\subsection{Enactment}

Automatic endorsement carries minimal time overhead.  As each block is produced, the corresponding  pool identifier needs to be
recorded.  The total stake that is prepared to upgrade needs to be calculated prior to the endorsement.  Assuming that there are
approximately \emph{2k} block producing pools and \emph{k} is 2,500 or less, this will involve a maximum of 2,500 additions,
which should have negligible performance impact.

\subsection{Voting Centre}

Supporting the voting centre adds no specific costs to the node.  All required capability is available through inspection of
the state of the chain, and does not require an active node connection.  A \texttt{db-sync} or other connection is sufficient
to carry out all necessary operations.
