%&pdfLaTeX
% !TEX encoding = UTF-8 Unicode
\documentclass{article}
\usepackage{ifxetex}
\ifxetex
\usepackage{fontspec}
\setmainfont[Mapping=tex-text]{STIXGeneral}
\else
\usepackage[T1]{fontenc}
\usepackage[utf8]{inputenc}
\fi
\usepackage{textcomp}

\usepackage{amssymb}
\usepackage{fancyhdr}
\renewcommand{\headrulewidth}{0pt}
\renewcommand{\footrulewidth}{0pt}
\usepackage{color}

\definecolor{color02}{rgb}{0.98,0.01,0.03}
\definecolor{color03}{rgb}{0.98,0.01,0.03}

\begin{document}

\textbf{Review of The Design of the Cardano Ledger with Automated Parameter Updates 
and Central Fund Transfers (April 30, 2021 version)}

\vspace{12pt}
\begin{center}
\textbf{Possible Current Issues: }
\end{center}

\vspace{12pt}
\baselineskip=12pt
\leftskip=0pt
- Some questions about stake snapshot versus vote delegation: p19, Fig 2: In the 
figure, the snapshot of the stake distribution is concurrent to the delegates voting. 
However, in the real execution the snapshot must be before the voting, right? Or 
else how could pools that are making blocks know whether a vote is valid and should 
be included in a block? From p30, Fig 7 the snapshot is before voting, which makes 
sense. However two questions:

\parindent=18pt
(1) Can delegations change during vote casting time, or once voting period begins 
delegation changes do not count? From Fig 7 it seems the latter is true. However, 
in Section 7.3 (Tallying Votes) it says ``total vote... is the sum of all active 
delegations at the point that the vote is tallied'', which feels it implies the 
former.

\vspace{12pt}
\parindent=0pt
{\color{color02} Delegations and vote choices can change at any time prior to the 
vote deadline.  We will adjust Figure 7.}

\vspace{12pt}
\parindent=18pt
(2) There may be edge cases to consider. For example: Suppose at the time of the 
stake snapshot I have some ada. However, after the snapshot I sell my ada. Should 
I still be able to redelegate my vote?

\vspace{12pt}
\parindent=0pt
{\color{color03} This is inevitable with a snapshot approach.  Likewise not being 
able to vote immediately that ada is acquired.  It is, of course, consistent with 
block production (I can earn rewards after I have sold my ada), so carries no greater 
risk.  The alternative (compute a snapshot for each vote) would be very inefficient 
(repeated snapshotting, which we need to avoid if possible), and also leads to 
``buy significant ada, vote, sell ada''.  Token locking helps solve that, but reduces 
participation, which is highly undesirable (and could reduce security/lead to lack 
of progress, depending on how totals were calculated).  Overall, the edge cases 
are likely to be minor, with less serious consequences than the alternative.}

\vspace{12pt}
- p23, Option 3 (transaction identifier is used to identify the proposal): This 
can be dangerous, since the location the proposal ends up on chain is not fixed 
until it is part of common prefix. If there are multiple concurrent proposals this 
can lead to people accidentally voting for the wrong proposal (or worse, malicious 
pools can exploit this for their advantage).

\vspace{12pt}
{\color{color02} Noted.   It would be unlikely, of course, but the proposal contains 
additional information that tie to a specific set of actions, so this could be 
hashed, for example.}

{\color{color02} Pools don't vote, but delegates could be malicious.}

\vspace{12pt}
- Section 8 and 9: It is not very clear right now whether multiple protocol version 
changes can happen in a single epoch.

\parindent=18pt
(1) Might there be issues if this can happen?

\vspace{12pt}
\parindent=0pt
{\color{color02} Only one could be adopted (the change can only be one Hamming 
distance from the current protocol version).  All minor version changes are equivalent, 
as are all major version changes,}

{\color{color02} so the only possible ambiguity comes where there is a conflict 
between a major and a minor upgrade.  The current design resolves this mechanically 
through temporal ordering, but an alternative would be for eg a major upgrade to 
override a minor upgrade (in which case the minor upgrade would never happen).}

\vspace{12pt}
\parindent=18pt
(2) If it can happen, Section 9.2 which says ``multiple update proposals...are 
enacted strictly in the order that they were submitted'' should also include ``and 
enacted in the order of the protocol version for protocol version changes''.

\vspace{12pt}
\parindent=0pt
{\color{color02} That is not necessary.  See above.  It probably also doesn't work 
technically (AFAIK, the hard fork combinator will only process one upgrade at an 
epoch boundary --- I will check its semantics!).}

\vspace{24pt}
\begin{center}
\textbf{Security concerns:}
\end{center}

\vspace{12pt}
\baselineskip=12pt
\leftskip=0pt
- p14, ``resolve conflicts between proposals automatically'' - There should be 
a lot of care taken when going this route, since a voter can vote for one proposal 
A without knowledge about another proposal B. The two proposals might not even 
conflict, but just not synergise well. For example, if A reduces the max size of 
a block and B increases the max size of a transaction, then together they may cause 
undesirable consequences (e.g. too few transactions allowed in a block).

\parindent=18pt
-\texttt{>} Perhaps a way to combat this is allow a proposal to declare a ``conflict 
set'', e.g. a proposal A is not compatible with another proposal B, if B modifies 
parameter x. 

\vspace{12pt}
\parindent=0pt
{\color{color02} This solution would require knowledge of all other proposals (which 
would be impractical), and would only work for a rational/honest submitter set. 
 If we assume a rational/honest submitter group, however, then we can assume that}

{\color{color02} they will anyway have resolved the conflict, in which case this 
is not really necessary.  The only solution I can envisage is to automatically 
check for inconsistencies, but that is difficult to achieve with the current parameter 
design (you would perhaps need logical constraints on each parameter to define 
possible conflicts).  Or we can rely on rational/honest voters to detect possible 
conflicts.}

\vspace{12pt}
\parindent=18pt
-\texttt{>} Another thing to be careful about is that temporal orderings might 
not be reliable, since the order of blocks can be (at least slightly) influenced 
by malicious parties.

\vspace{12pt}
\parindent=0pt
{\color{color02} Noted.  We can assume that the submitter group is honest and aware 
of possible malicious influence.}

\vspace{12pt}
- p15, Security Requirements: Another issue that would be good to address is that 
it seems stake pools have a big power over voting. For example, suppose there is 
a proposal that is desirable for normal users of the system, except many stake 
pools do not like this proposal. Then, it may be possible for these stake pools 
to ensure that this proposal is never put onto the blockchain, or at least delay 
this proposal until e.g. the vote deadline. 

\parindent=18pt
-\texttt{>}  Of course, by the security assumption, honest stake pools will include 
even proposals they do not like, but this may not be the case for rational stake 
pools.

\vspace{12pt}
\parindent=0pt
{\color{color02} To be clear, stake pools endorse, but don't vote or submit proposals. 
 The vote and endorsement deadlines are absolute, so there can be no delay (a proposal 
is either accepted at the deadline or it is not - everything else is irrelevant).}

{\color{color02} Endorsement is limited to protocol upgrades, so pools have no 
direct control over parameter updates - submitters and delegators have all the 
power.  }

\vspace{12pt}
{\color{color02} A pool that doesn't endorse a protocol upgrade will become disconnected 
from the main chain.  So it's important to avoid chain splits that sufficient pools 
endorse a proposal (as defined by the endorsement threshold).  }

{\color{color02} There is no way to force a pool to upgrade.  So yes, they do collectively 
have power of veto on protocol upgrades.  That is seen as reasonable - they bear 
the cost of maintaining the network}

{\color{color02} and have technical knowledge of the impact of an upgrade.  However, 
if the upgrade was in the interest of the chain, and stakeholders agreed on this, 
they could choose to re-delegate their stake}

{\color{color02} to pools that would upgrade (even forming new pools).  So no group 
of pools could postpone an honest upgrade indefinitely (even 100\% ``dishonesty'' 
can be overcome by forming new pools).}

\vspace{12pt}
{\color{color02} In practice, pools are unlikely to oppose an upgrade unless there 
are technical issues with it.}

\vspace{12pt}
- p18, Submitters: For decentralisation it is important to ensure that submitters 
is not restricted to too few members.

\vspace{12pt}
{\color{color02} Noted and agreed.  There will be a practical limit on numbers 
(how many signatures can be included in a transaction), but this will be around 
200-400.}

{\color{color02} There is a requirement to limit the size of the submitter group 
to avoid ``flooding'' (a denial of service where honest proposals are swamped by 
dishonest ones, so voters are overwhelmed by choice/cost).}

\vspace{12pt}
- p22, ``Central Funds Transfer Body'' - Since there is no UTxO with the treasury/reserves, 
one must be careful about accounting to avoid double-spending. 

\vspace{12pt}
{\color{color02} Noted.  There are existing formal accounting rules/properties 
that are followed for manual funds transfers, including overall preservation of 
ada.  These need to be preserved.}

\vspace{12pt}
- p23, comment on using multisig for submitter signing: Depending on the size of 
the submitter group, one must be careful about the signatures not exceeding max 
transaction size.

\vspace{12pt}
{\color{color02} Yes.  This limits the number of signatories.  The same issue applies 
to a specialised proposal format.}

\vspace{24pt}
\begin{center}
\textbf{Questions/Comments/Suggestions:}
\end{center}

\vspace{12pt}
\baselineskip=12pt
\leftskip=0pt
- p12, section 1.4 - Here can also say that exchanges/proxy holders can declare 
themselves as not holding any stake (their address should clearly indicate this)

\vspace{12pt}
{\color{color02} At present they don't.  There is an address format for this purpose, 
but it is unused, and cannot be enforced.}

\vspace{12pt}
- p21, the comment on submitters being trustworthy: Perhaps we can use a similar 
method to choosing slot leaders to choose the submitters, so we use the same assumption 
of honest majority of stake as the cryptographic security.

\vspace{12pt}
{\color{color02} Full randomness doesn't work for this purpose.  A submitter has 
to be prepared to submit a proposal, to collate counter-signatures, and to pay 
any fees.}

\vspace{12pt}
- p21, the comment on setting minimal thresholds for vote enactment''. I agree 
there must be some kind of limit of how low a vote threshold can be, or else there 
could be proposals that are enacted even if very few people voted on it. An idea 
is to require that a proposal must *both* (1) have a number of ``yes'' votes that 
is above the vote threshold, and (2) have a number of ``no'' votes that is lower 
than the vote threshold. This means a proposal with low threshold can be outvoted 
if a problem is found with it, and also makes use of the ``no'' vote (by Section 
7.1, at the moment, there is no point of casting a ``no'' vote as this is the same 
as not voting.

\vspace{12pt}
- p23, comment on collision resistance: Yes, collision resistance is required.

\vspace{12pt}
{\color{color02} Thank you.}

\vspace{12pt}
- p25, the ``Confirm.'' and ``Confirm this.'' comments: Should be ``yes'' to both.

\vspace{12pt}
{\color{color02} Thank you.}

\vspace{12pt}
- p31, comment on recording the tally on chain: Yes, I believe it could be good 
for performance, similar to recording the result of scripts.

\vspace{12pt}
{\color{color02} Thank you.  We will note this.}

\vspace{12pt}
\begin{center}
\textbf{Typos/Minor:}
\end{center}

\baselineskip=12pt
\leftskip=0pt
{\color{color02} Thank you. We will apply these.}

\vspace{12pt}
- p4 ``There c'' (unfinished sentence)

\vspace{12pt}
- p9 ``nomrmal'' -\texttt{>} ``normal''

\vspace{12pt}
- p19 ``where the proposed...'' -\texttt{>} ``where the proposal...''

\vspace{12pt}
- p23 ``All update'' -\texttt{>} ``All updates''

\vspace{12pt}
- p27 Missing citation at ``...from the Shelley delegation design document...''

\vspace{12pt}
- p28 ``...refer to a vote credential that is must be...'' -\texttt{>} ``refer 
to a vote credential that must be...''

\vspace{12pt}
- p28 Missing citation at ``...stake address references (see ?)...''

\vspace{12pt}
- p34 Missing citation in Section 9.1 after ``hard fork combinator''

\vspace{12pt}
- p35 Missing descriptions in table

\vspace{12pt}
- p36 ``prpgress'' -\texttt{>} ``progress''

\newpage

\end{document}
