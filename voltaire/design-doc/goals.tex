\section{Detailed Design Goals}
\label{sect:goals}

The primary goals of this design are:

\begin{itemize}
\item
  \textbf{Decentralised Governance:}
  Support a decentralised governance mechanism for Cardano.
\item
  \textbf{Genesis Key Retirement:}
  Remove the dependence of the Cardano blockchain on a small set of genesis keys/delegates.
\item
  \textbf{Security:}
  Maintain the security of the Cardano network.
\item
  \textbf{Progress:}
  Ensure that the blockchain protocol can evolve in the necessary way.
\item
  \textbf{Smooth Transition:}
  Evolve smoothly from the current federated governance
\end{itemize}

Where these goals conflict, it is, of course, absolutely essential to maintain the security of the network.  Maintaining progress/enabling
evolution is also important, since otherwise governance decisions will be ineffectual/not enacted.

\subsection{Overall Governance Goals}

The primary reason for decentralisation is to ensure long-term compliance with emergent and evolving international financial regulations.
Secondary reasons include improved community engagement, alignment of decision-making authority with increased responsibility for maintaining the blockchain,
ensuring the long term stability of the Cardano blockchain.  Decentralisation means that decisions about the operation and evolution of the blockchain
should be taken collectively by all stakeholders.
%
Achieving these goals requires that:

\begin{itemize}
\item
  The governance process is transparent -- the progress of decisions can be monitored and any deviations are properly explained to stakeholders.
\item
  All opinions are considered.
\item
  Control of the blockchain rests with those who have the greatest direct stake in the operation of the blockchain as holders of ada or as maintainers of the blockchain.
\item
  Control of the blockchain is not unduly concentrated -- small, unrepresentative groups of powerful individuals should not be able to take over the operation
  of the blockchain.
\item
  The blockchain is robust to change -- it should be possible to recover from an incorrect decision, and changes do not expose the blockchain to attack.
\item
  The mechanisms that are implemented properly incentivise stakeholders to take decisions that are in the long-term interest of the blockchain.
\item
  There are checks and balances that ensure that technical knowledge and advice is properly considered as part of the decision making and implementation process.
  This is especially important where there are security concerns.
\item
  There is traceable continuity in the blockchain -- there is an on-chain record of all the changes that are made, including the on-chain voting outcomes.
\item
  The transition path to decentralised governance is clearly identified and explained, with clear steps and gates.
\end{itemize}


It is, of course, not necessary (or probably even sensible) for every aspect of
the process to be fully decentralised, only that the ultimate \emph{control} of
the blockchain is decentralised to the extent that is required for effective
governance.  This enables even those who are less technically expert to participate in
key governance decisions.

\pagebreak
\subsection{On-Chain Requirements}

There are many legitimate ways to meet the off-chain governance requirements.  Since this document focuses primarily on the on-chain technical design requirements, these will not be covered here.

\paragraph{Mandatory Requirements}.  The on-chain mechanism must:

\begin{itemize}
\item
  allow changes to be made to any updatable protocol parameter, in any legal way;
\item
  allow major/minor protocol versions to be upgraded;
\item
  allow the fulfilment of all kinds of central funds transfers;
\item
  not require the use of genesis keys or delegates in any part of the protocol;
\item
  allow separate per-proposal deadlines to be set for voting and endorsement;
\item
  ensure that deadlines are reasonable;
\item
  allow an epoch to be chosen for enactment of a proposal;
% \item
%   be able to faithfully enact the governance decisions that have been taken off-chain.
\item
  be secure;
\item
  be efficient;
\item
  meet the Cardano blockchain stability window requirements;
\item
  allow delegation of voting rights from ada holders to a dedicated voter group;
\item
  assign voting and endorsement rights in proportion to the amount of ada that is held;
\item
  check that sufficient stake pools have upgraded, where the proposal is to change upgrade the protocol version;
\item
  prevent denial-of-service attacks by allowing large/unlimited numbers of proposals to be submitted;
\item
  automatically enact proposals at the stated epoch boundary, provided that the necessary voting and endorsement thresholds have been met.
\end{itemize}

\paragraph{Optional Requirements}.  In addition the on-chain mechanism should:

\begin{itemize}
\item
  allow multiple proposals to be in-flight and enacted in a single epoch;
\item
  resolve conflicts automatically (e.g. using temporal ordering, or some other consistent and predictable mechanism);
\item
  (perhaps, allow different thresholds to be set for different kinds of proposal);
\item
  as far as possible, be consistent with the existing manual update mechanism.
\end{itemize}
