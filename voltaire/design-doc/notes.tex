\pagebreak
\section*{Design Notes}

This section covers some key design issues that have been discussed, and explains how they have been resolved.

\newcounter{issue}

\stepcounter{issue}
\subsection*{Issue \theissue{}: Should votes be based purely on stake?}

\subsubsection*{Description of Issue.}

Basing voting power purely on the stake that is held potentially disenfranchises some groups that have a
significant investment in the ecosystem.  While block production relies on \emph{Proof of Stake},
it is not necessarily clear that voting should follow the same principle.

\subsubsection*{Alternative Solutions.}

\begin{enumerate}
\item
  \emph{Votes are based on stake.}  This is the obvious solution.
\item
  \emph{Votes are based on fees paid.}  This enfranchises the users of the system.
\item
  \emph{Votes are based on transaction volume.}  This also enfranchises users.
\item
  \emph{Votes are based on identity.}  This creates a more democratic system.
\item
  \emph{Hybrid.}  For example, part of the vote is based on stake, part on the fees paid, in some pre-determined proportion.
\end{enumerate}

\subsubsection*{Chosen Resolution.}

We will use the obvious solution (Option 1: stake-derived voting).  This fits
with the existing stake delegation design, so simplifying the design and
implementation, and minimising the number of new mechanisms that are required.
It also simplifies security concerns -- it is not possible for a cabal to take
over the blockchain, without first gaining the authorisation of the majority of
the held stake.

\stepcounter{issue}
\subsection*{Issue \theissue{}.}

\subsubsection*{Description of Issue.}

\subsubsection*{Alternative Solutions.}

\subsubsection*{Chosen Resolution.}

\pagebreak
