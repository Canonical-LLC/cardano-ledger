\pagebreak
\section*{Design Notes}

This section covers some key design issues that have been discussed, and explains how they have been resolved.

\newcounter{issue}

\stepcounter{issue}
\subsection*{Issue \theissue{}: Should votes be based purely on stake?}

\subsubsection*{Description of Issue.}

Basing voting power purely on the stake that is held potentially disenfranchises some groups that have a
significant investment in the ecosystem.  While block production relies on \emph{Proof of Stake},
it is not necessarily clear that voting should follow the same principle.

\subsubsection*{Alternative Solutions.}

\begin{enumerate}
\item
  \emph{Votes are based on stake.}  This is the obvious solution.
\item
  \emph{Votes are based on fees paid.}  This enfranchises the users of the system.
\item
  \emph{Votes are based on transaction volume.}  This also enfranchises users.
\item
  \emph{Votes are based on identity.}  This creates a more democratic system.
\item
  \emph{Hybrid.}  For example, part of the vote is based on stake, part on the fees paid, in some pre-determined proportion.
\end{enumerate}

\subsubsection*{Chosen Resolution.}

We will use the obvious solution (Option 1: stake-derived voting).  This fits
with the existing stake delegation design, so simplifying the design and
implementation, and minimising the number of new mechanisms that are required.
It also simplifies security concerns -- it is not possible for a cabal to take
over the blockchain, without first gaining the authorisation of the majority of
the held stake.

\stepcounter{issue}
\subsection*{Issue \theissue{}: How to prevent proxy Ada holders from exercising voting rights.}

\subsubsection*{Description of Issue.}

Proxy holders, such as exchanges, may exercise voting rights.  This may affect legitimacy: votes are exercised by those who do not own Ada.

One solution that has been proposed is ``token locking''.  Votes are
only counted for tokens that have been locked.  Locked tokens cannot
be sold or transferred.  This discourages proxy holders from voting by
restricting token liquidity.  However, experience with the Catalyst
voting system shows that this also reduces normal voter participation,
with a consequent effect on representation and
legitimacy. \todo{Can we quantify this?}
A second solution is to provide a second kind of address that is designed to be used specifically by proxy holders (``enterprise addresses'')
and that restricts voting rights.
% A significant proportion of Ada holders are discouraged from voting.

\subsubsection*{Alternative Solutions.}

\begin{enumerate}
\item
  \emph{Only locked tokens may vote.}  Unfortunately, this also reduces voting participation by direct Ada holders, who must accept a loss of liquidity in return for voting rights.
\item
  \emph{``Enterprise'' addresses do not possess voting rights.}  Proxy holders register enterprise addresses, which forgo voting (and stake delegation) rights.
\item
  \emph{There is no explicit control over proxy voting rights.}  Ada holders may enter a legal agreement with their proxies, electing to retain voting
  rights but to delegate staking rights, for example.  
\end{enumerate}

Although both token locking and enterprise addresses have been implemented, neither has been effective.
The rewards mechanism incentivises proxy holders to delegate stake (meaning that enterprise addresses are rarely, if ever, used),
and token locking creates risk for normal ada holders, so reducing participation.


\subsubsection*{Chosen Resolution.}

Since neither token locking nor enterprise addresses have been shown to be effective, we have adopted the third approach (no explicit control)
as a pragmatic solution, pending further investigation.
Any tokens may vote: tokens do not need to be locked for voting purposes.  Enterprise addresses are supported (and have no voting rights), but
there is no on-chain mechanism to enforce this.

\pagebreak
