\pagebreak
\section{User Experience}
\label{sect:ux}

This appendix outlines how the voting experience could appear to ada holders and other user groups.

\subsection{General Ada Holder Experience}

\paragraph{General Process.} An ada holder registers their account in the voting centre.  As a result, they obtain both Catalyst voting rights and on-chan vote delegation rights.
The voting centre displays a list of registered delegates, and allows them to delegate their vote to any delegate that they choose.  Transaction
fees are paid for registration and for vote delegation.  The ada holder may choose to participate in the off-chain discussion and voting process for any
parameter update or funds transfer. They will receive ada rewards for their participation.
\footnote{Voting rights are proportional to the ada that is held, both in Catalyst and on-chain.  However, there may be lag in the Catalyst system.
  so that the Catalyst system could be kept up to date.  At present, the stake transfer is manual and uses a \texttt{db-sync} snapshot -- this would need to be automated, and there might need to be some changes to the Jormungandr system
  so that it could easily record dynamic stake changes.  } % Token locking information might also need to be recorded.}

\paragraph{Catalyst Process.}

This needs to be defined, but in essence:

\begin{itemize}
\item
  Ada holders register for Catalyst voting through the voting centre;
\item
  Voters are informed of active proposals in areas that interest them\footnote{There is a tension here between participation and overload -- ada holders may find parameter updates to be too technical, for example};
\item
  Using the Catalyst vote mechanism, ada holders vote on whether the proposal should be accepted, rejected or revised -- there may be multiple voting rounds
  before a decision is finalised;
%  \khcomment{This may reflect a change in the current Catalyst process?}
\item
  Once a proposal is accepted, it proceeds to formal implementation and on-chain enactment.
\end{itemize}

\paragraph{On-Chain Vote Delegation.}

Ada holders may delegate their on-chain vote whenever they choose.  Vote delegation may be changed at any point in time through the voting centre, in
exactly the same way as stake pool delegation.  Ada holders are warned if their chosen delegate announces their retirement, and may then redelegate their vote if they
wish.  If they do not redelegate their vote, it will not count towards any future decision.\khcomment{Note the different possible voting thresholds.}
When an ada holder withdraws their ada holding, they will lose their voting rights with effect from the next stake snapshot (i.e. at the start of the next
epoch).  Similarly, when they acquire ada, they will acquire new voting rights with effect from the next stake snapshot.
Ada holders may track how their delegates have voted on-chain, and may reassign their vote delegation if they wish.

\subsection{Delegate Experience}

Ada holders may choose to become vote delegates.  They may do this on their own behalf, or as public delegates that act as proxies for other ada holders.
Public delegates will be influential community members, with public duties and responsibilities.
A delegate first registers themselves in the voting centre.  As public delegates, they may then advertise their voting intentions to other ada holders and canvas for vote delegations.
They will participate in and follow discussions about proposals in various forums, track on-line proposal submissions, and
record their vote for each proposal before the stated deadline using the voting centre.  They may track the outcome of each proposal using the proposal explorer, and inform their
delegators of the proposal outcome.
\khcomment{It is not obvious what motivates delegates to participate other than a sense of public duty? They have duties and responsibilities, but no immediate rewards.}

\subsection{New Tools/Mechanisms to Support the User Experience}

A number of new tools or mechananisms are needed to support the user experience:

\begin{enumerate}
\item
  A way to prepare business for discussion and voting via Catalyst;
\item
  A way to track proposals -- this implies either a significant extension to the voting centre, or some new ``proposal explorer'' capability;
\item
  Changes to the voting centre to support on-chain vote registration, vote delegation etc.;
\item
  A way for delegates to register and to record their on-chain votes -- perhaps via the voting centre, or perhaps via CLI commands;
\item
  A way for delegates to make themselves known to potential delegators;
\item
  A way for ada holders to become aware of delegates, including the number of votes that they control.
\end{enumerate}

% All of these mechanisms will need to be designed and implemented.

Proposal submission and endorsement will be done by technical experts, so can be handled via CLI commands.  It may be reasonable for delegates
to also use the CLI, but they may have less technical expertise than stake pool operators, for example.

\paragraph{Tracking Proposals.}

An explorer or other mechanism needs to be set up so that the status and progress of proposals can be tracked by stakeholders.  A single off-chain proposal may give rise
to multiple on-chain proposals that need to be linked to the original proposal.  Section~\ref{sect:proposalid} describes some ways that proposals may be identified, and used
to link off-chain and on-chain proposals.
