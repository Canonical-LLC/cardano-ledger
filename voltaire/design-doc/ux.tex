\pagebreak
\section{User Experience}
\label{sect:ux}

This appendix outlines how the voting experience could appear to

\subsection{General Ada Holder Experience}

\paragraph{General Process.} An ada holder registers their account in the voting centre.  As a result, they obtain both Catalyst voting rights and on-chan vote delegation rights.
The voting centre displays a list of registered delegates, and allows them to delegate their vote to any delegate that they choose.  Transaction
fees are paid for registration and for vote delegation.  The ada holder may choose to participate in the off-chain discussion and voting process for any
parameter update or funds transfer. They will receive ada rewards for their participation.
Voting rights are proportional to the ada that is held, both in Catalyst and on-chain.  However, there may be lag in the Catalyst system.\khcomment{It would be possible to export the vote snapshot at each epoch,
  so that the Catalyst system could be kept up to date.  At present, I think the stake transfer is manual -- this would need to be automated, and there might need to be some changes to the Jormungandr system
  so that it could easily record the changes.}

\paragraph{Catalyst Process.}

This needs to be defined, but in essence:

\begin{itemize}
\item
  Ada holders register for Catalyst voting through the voting centre;
\item
  Voters are informed of active proposals in areas that interest them;
  \khcomment{There is a tension here between participation and overload -- ada holders may find parameter updates to be too technical, for example}
\item
  Ada holders vote on whether the proposal should be accepted, rejected or revised -- there may be multiple voting rounds
  before a decision is finalised;
  \khcomment{This may reflect a change in the current Catalyst process?}
\item
  Once a proposal is accepted, it proceeds to formal implementation and on-chain enactment.
\end{itemize}

\paragraph{On-Chain Vote Delegation.}

Ada holders may delegate their on-chain vote whenever they choose.  Vote delegation may be changed at any point in time throgh the voting centre, in
exactly the same way as stake pool delegation.  Ada holders are warned if their chosen delegate announces their retirement, and may then redelegate their vote if they
wish.  If they do not redelegate their vote, it will not count towards any future decision.\khcomment{Note the different possible voting thresholds.}
When an ada holder withdraws their ada holding, they will lose their voting rights with effect from the next stake snapshot (i.e. at the start of the next
epoch).  Similarly, when they acquire ada, they will acquire new voting rights with effect from the next stake snapshot.

\paragraph{Tracking Proposals.}

An explorer or other mechanism needs to be set up so that the status and progress of proposals can be tracked.  A single off-chain proposal may give rise
to multiple on-chain proposals that need to be linked.

\subsection{Delegate Experience}

Ada holders may choose to become vote delegates.  They may do this on their own behalf, or as public delegates.  Public delegates


\subsection{Additional Tools}

A number of new tools or mechananisms are needed:

\begin{enumerate}
\item
  A way to prepare business for discussion and voting via Catalyst;
\item
  A way to track proposals -- this implies either a significant extension to the voting centre, or some new ``explorer'' capability;
\item
  Changes to the voting centre to support on-chain vote registration, delegation etc.
\item
  A way for delegates to register and to record their on-chain votes -- perhaps via the voting centre, or perhaps via CLI commands;
\item
  A way for delegates to make themselves known to potential delegators.
\item
  A way for ada holders to become aware of delegates.
\end{enumerate}

All of these mechanisms will need to be designed and implemented.  They mainly involve the Adrestia and Daedalus teams.

Proposal submission and endorsement will be done by technical experts, so can be handled via CLI commands.  It may be reasonable for delegates
to also use the CLI, but they may have less technical expertise than stake pool operators, for example.
