\section{Comparison with the approach taken by the Priviledge Project}
\label{sect:priviledge}

The EU Priviledge project aims to produce a blockchain-agnostic fully decentralised voting system.
Its design is described in~\ref{Priviledge}, and has been used to inform the design that is described here.
The PUP design limits itself to on-chain submission, voting and enactment of proposals, using an alternative mechanism
for ideation, dicussion, and general voting.  In contrast, the Priviledge design also covers on-chain ideation and voting.
The key differences between the PUP mechanism and the one that has been produced for the EU Priviledge project are:

\begin{tabular}{||p{3in}|p{3in}||}
  \hline\hline
  \textbf{Priviledge} & \textbf{PUP}
  \\\hline
  Completely on-chain & Uses off-chain processes as well as on-chain ones \\\hline
  Proposal implementations must be completed before any voting can commence\khcomment{Confirm this}. & Proposal implementations must be completed before being submitted on-chain \\\hline
  New end-to-end mechanism & Conservative automation of existing update mechanism \\\hline
  Does not restrict proposal submission to specific individuals & Restricts on-chain submission to a specific group (aims to avoid ``flooding''/denial-of-service) \\\hline
  Proposals are tracked throughout & Proposals are tracked once they are on-chain, mechanisms are provided to tie on-chain to off-chain proposals \\\hline
  Proposals are completely independent & A single off-chain proposal may initiate multiple on-chain proposals for enactment \\\hline
  Security-related proposals follow the standard process\khcomment{Confirm this.}  & Security-related proposals may bypass some off-chain processes \\\hline
  A single vote may only be used for one active proposal\khcomment{Confirm this.} & A single vote may be used for any active proposals \\\hline
  Does not identify different actor groups & Clearly identifies different actor groups: submitters, delegates etc. \\\hline
  Supports explicit vote abstention & Treats abstention as rejection \\\hline
  Allows votes to be removed prior to tally & Allows votes and delegations to be changed prior to tally (equivalent effect) \\\hline
  Allows multiple voting rounds for each proposal & Allows only one on-chain voting round for a proposal \\\hline
  Allows  vote stake snapshots to be taken at arbitrary points & Takes vote snapshots at fixed points \\\hline
  Explicit proposal prioritisation -- order of enactment can be specified  & Submission-based prioritisation -- proposals are enacted in order of submission \\\hline
  Proposals can be cancelled & Proposals cannot be cancelled once they are approved and endorsed \\\hline
  Includes software upgrades (not just protocol upgrades) & Does not consider software upgrades, except where these are protocol version changes \\\hline
  Does not handle central funds transfers & Handles central funds transfers \\\hline
  Does not support vote delegation\footnote{New work is in progress.} & Supports vote delegation \\\hline
  Does not consider Cardano blockchain stability windows & Explicitly considers Cardano blockchain stability windows \\\hline
  Requires version numbering in each proposal & No version numbering required  \\\hline
  Single proposal enactment at an epoch boundary\khcomment{From memory.  Check this.}  & Multiple proposal enactment at each epoch boundary \\\hline
  \hline
\end{tabular}

Most of these differences reflect either differences in stated user requirements, or a desire to make
the PUP mechanism generally consistent with the existing update mechanism.
Other differences are simplifications that are aimed at improving peformance or security (e.g. vote
stake snapshots, restricting submission) or reducing development and testing time (e.g. version numbering, time-based prioritisation).
Most of the Priviledge mechanisms could be added at a later date, if required, assuming that suitable adaptations were carried out.
The proposal submission process could then be adapted so that proposals were automatically carried forwards from previous Priviledge stages.
